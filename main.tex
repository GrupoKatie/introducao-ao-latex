\documentclass{article}
\usepackage[utf8]{inputenc}
\usepackage[brazilian]{babel}
\usepackage{url}
\usepackage{hyperref}
\hypersetup{
    colorlinks=true,
    linkcolor=black, %sumário
    filecolor=magenta,      
    urlcolor=magenta,
}
\usepackage{graphicx}
\usepackage{indentfirst}
\usepackage{fancyhdr}
\pagestyle{fancy}
\lhead{Introdução ao \LaTeX}
\rhead{Grupo Katie}

\title{Introdução ao \LaTeX}
\author{Grupo Katie \\ katie@ic.ufal.br}
\date{6 de Março de 2020}

\begin{document}

\maketitle
\begin{figure}[ht]
    \centering
    \includegraphics[scale=0.3]{logo.png}
\end{figure}
\thispagestyle{empty}
\newpage
\tableofcontents
\thispagestyle{empty}
\newpage

\section{Introdução}
Querida(o) participante, este é o seu primeiro \LaTeX file. Bem-vinda(o) ao curso de Introduçao  ao {\LaTeX} ministrado pelo \textit{Grupo Katie}.

\subsection{Como funciona o \LaTeX?}
O {\LaTeX} utiliza \textit{packages}, que são pacotes contendo várias funções. Tais funções representam as funcionalidades do {\LaTeX} para facilitar nosso trabalho. Daí, você escreve um pedaço de código e compila.

\subsubsection{Mas o que é compilar?}
Compilação é, basicamente, um processo que consiste em transformar o código que você escreveu em um arquivo do tipo \textit{pdf}. Mas não se preocupe com isso, pois esse processo é feito pelo próprio \textit{Overleaf} com apenas um clique.

Pressione CTRL + Enter e veja o que acontece. Interessante, não? Vamos adiante.

\subsubsection{Como funcionam os blocos de código?}
As funções do {\LaTeX} seguem a seguinte estrutura:
\begin{center}
$\backslash begin\{document\}
\cdots 
\cdots 
\cdots
\backslash end\{document\}$
\end{center}

Algumas características:
\begin{itemize}
    \item Todos os comandos iniciam com $\backslash$ (contra barra);
    \item Todo documento inicia-se com o comando \textit{$\backslash$documentclass};
    \item O argumento entre as chaves (em inglês \textit{curly braces}) dita o nome da função que você vai utilizar;
    \item O símbolo de porcentagem \% inicia um comentário no código e todo o conteúdo após esse símbolo será ignorado pelo compilador.
\end{itemize}

\subsection{Material de apoio}
O \LaTeX, o qual não é uma linguagem de programação, é simples de manusear no início e vai complicando com o tempo, vamos ver até onde conseguimos chegar juntos!

Para descobrir como se faz qualquer coisa em \LaTeX, basta pesquisar no Google. Mas sem desespero, compartilhamos uma \href{https://drive.google.com/open?id=1iMswdfMALd7LLEmVAkGpLoclAJe6kZuR}{pasta} no Google Drive contendo apostilas de \LaTeX, as quais utilizamos com frequência, na intenção de guiar você.

Está com preguiça de abrir o \textit{Drive}? Nós entendemos! Clique \href{https://www.ime.usp.br/~viviane/MAP2212/minicurso.pdf}{aqui} para conhecer uma apostila incrível do Instituto de Matemática e Estatística da USP.


\begin{itemize}
    \item Não precisa ter pressa para aprender, faça no seu tempo;
    \item Em caso de dúvidas, só chamar;
    \item Não se limite às apostilas que compartilhamos com você, pode-se e deve-se sempre pesquisar mais;
    \item Pesquisar em inglês traz resultados mais consistentes na maioria das vezes.
\end{itemize}

\subsection{Utilização do GitHub}
O GitHub é uma empresa que proporciona armazenamento e versionamento de códigos e projetos em geral. Caso você não conheça o \href{https://github.com/}{GitHub}, peça ajuda a algum(a) monitor(a) para fazer o seu cadastro. Você também pode conhecer o GitHub do Grupo Katie \href{https://github.com/GrupoKatie}{aqui}.

Com o fito de manter um controle sobre o quanto você tem evoluído, peço encarecidamente que você gere um \textit{pdf} a cada vez que conseguir algo novo e diferente, pois pode ser que nós não conheçamos ainda a ferramenta que você utilizou, afinal não sabemos de tudo, né? Vá guardando esses arquivos em uma pasta do GitHub.

Sim, é possível fazer upload do código para uma pasta do GitHub e eu recomendo fortemente que o faça! Deixe a pasta pública e compartilhe o link conosco para que possamos acompanhar seu desempenho.

Divirta-se, isso é como programar :)

\subsection{Observações Importantes}

Recomendamos que você compile o código a cada mudança significativa que fizer. Às vezes você poderá cometer um erro que impeça o compilador de gerar o arquivo, então ele fica todo vermelho raivoso e desesperador... TUDO BEM, acontece. Aquele \textit{CTRL+z} resolve temporariamente o problema até você encontrar o seu erro ou substituir aquela parte do código por outra mais eficiente.

Existem vários templates aqui no Overleaf, pode explorar à vontade, mas a ideia é construir o seu próprio arquivo, assim você aprende mais.

Sua presença é muito importante para nós :)

\newpage

\section{Primeira atividade}
A melhor forma de você aprender é praticando. Lembre-se que dispomos de monitores e monitoras para auxiliar você a qualquer momento, sinta-se à vontade. Inclua subseções e/ou subsubseções a seu gosto para manter a organização do arquivo. Então, mãos à obra!

\begin{enumerate}
    \item Inicie um novo documento na sua conta do \textit{Overleaf} contendo suas informações. 
        \begin{enumerate}
            \item Inclua um sumário e faça com que o arquivo seja enumerado apenas a partir da página seguinte ao sumário.
            
            \item Utilize a função \textit{enumerate} ou \textit{itemize} para citar 3 razões por você ter escolhido o seu curso de graduação.
        \end{enumerate}
         Dica: \url{https://www.latex-tutorial.com/tutorials/table-of-contents}
    
    \item Crie uma seção chamada \texttt{Matemática} e nesta:
        \begin{enumerate}
            \item Insira uma matriz 2x2 e uma matriz 3x4 de números binários aleatórios.
            \item Insira a fórmula da Equação de Gravitação Universal de Newton.
            \item Insira a Identidade de Euler.
            \item Insira uma identidade trigonométrica.
        \end{enumerate}
    Dica: \url{https://www.codecogs.com/latex/eqneditor.php}    
    
    \item Crie uma seção chamada \textsc{Imagens} e:
    \begin{enumerate}
        \item Insira uma foto da sua cientista favorita. Preste atenção na legenda da foto. 
        
        \item Insira um conjunto de fotos que ilustrem as quatro fases da Lua. Utilize a função \textit{subfigure} e preste atenção nas sublegendas. Você vai precisar do pacote \textit{subcaption}.
        \end{enumerate}
        Dica: \url{https://www.latex-tutorial.com/tutorials/figures/}
    
    
    
    \item Busque um texto do Google contendo a definição de computação, cole-o aqui e utilize funções de \textit{hyperlink} para referenciá-lo.
    
    Dica: \url{https://www.overleaf.com/learn/latex/Hyperlinks}
    
    \item Insira uma tabela contendo 3 elementos da tabela periódica, escolhidos por você, com seus respectivos valores de massa atômica e número atômico.
    
    Dica: \url{https://www.tablesgenerator.com/}
\end{enumerate}


\end{document}
